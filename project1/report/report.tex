\documentclass{article}
\usepackage[utf8]{inputenc}
\usepackage[
  a4paper,
  top=.8in,
  right=1.5in,
  bottom=1in,
  left=1.5in
  % margin=1in,
]{geometry}
\usepackage{hyperref}
\usepackage{amsthm}
\usepackage{amssymb}
\usepackage{amsmath}
\usepackage[parfill]{parskip}
% \usepackage{mathtools}
% \usepackage{csquotes}
\hypersetup{
  colorlinks=true,
  linkcolor=blue,
  filecolor=magenta,
  urlcolor=blue,
  pdfpagemode=FullScreen, 
}
\begingroup
  \makeatletter
  \@for\theoremstyle:=definition,remark,plain\do{
    \expandafter\g@addto@macro\csname th@\theoremstyle\endcsname{
      \addtolength\thm@preskip\parskip
      }
    }
\endgroup

\newcommand{\floor}[1]{\lfloor #1 \rfloor}
\newcommand{\ceil}[1]{\lceil #1 \rceil}

\newtheorem{theorem}{Theorem}[section]

\theoremstyle{definition}
\newtheorem{definition}{Definition}[section]

\theoremstyle{definition}
\newtheorem{remark}{Remark}[section]

\theoremstyle{definition}
\newtheorem{example}{Example}[section]

\begin{document}

\title{EDIN01 -- Project 1}
\author{Tony Jin (to1643ji-s@student.lu.se)}
\date{\today}
\maketitle

% ===== SECTION 1 =====

\section{Exercise 1}
Let $N$ be a 25 digit number and $\lfloor \sqrt{N} \rfloor$ is a 12 digit number.

The 12 digit is obviously bounded by $10^{11} \leq \floor{\sqrt{N}} < 10^{12}$, where $\floor{\sqrt{N}} \in \mathbb{N}$.
Thus, the seconds needed to factor the 25 digit number is a positive integer $t$ bounded by $$10^{4} \leq t < 10^{5},$$ as $10^{11} \cdot 10^{-7} = 10^{4}$ and $10^{12} \cdot 10^{-7} = 10^{5}$.

\section{Exercise 2}
We know that $\pi(10^{11}) = 4118054813$ and $\pi(10^{12}) = 37607912018$ \cite{prime}. Our new bounds for $t$ is then $$4118054813\cdot 10^{-7} \leq t < 37607912018\cdot 10^{-7}.$$

Assuming that every integer is 12 digts number with an equivalent binary representation of 40 bits, the total amount of memory needed is $164722192520$ bits $\approx 20$ gigabyte. A student budget is enough.

\section{Exercise 3}
The prime facrtors for $253808893609854792191119$ are $p = 496469737391, q = 511227320609$.

The quadratic sieve program below is written in Go.

\begin{verbatim}
package main

import (
  "fmt"
  "math"
  "math/big"
  "log"
  "os"
  "strings"
  "reflect"
  "strconv"
  "os/exec"
  "bufio"
)

// Big pi multiplication
func mult(factors []*big.Int) *big.Int {
  product := big.NewInt(1)
  for i := 0; i < len(factors); i++ {
    product.Mul(product, factors[i])
  }

  return product
}

// Return all primes less than `value`
func Factorbase(value int) []*big.Int {
  var primes []*big.Int
  f := make([]bool, value)
  for i := 2; i <= int(math.Sqrt(float64(value))); i++ {
    if f[i] == false {
      for j := i * i; j < value; j += i {
        f[j] = true
      }
    }
  }
  for i := 2; i < value; i++ {
    if f[i] == false {
      primes = append(primes, big.NewInt(int64(i)))
    }
  }

  return primes
}

// Check if a number `n` is B-smooth over a factorbase F
func FactorOverF(F []*big.Int, n *big.Int) ([]int, []*big.Int, error) {
  nCopy := new(big.Int).Set(n)
  var a []*big.Int

  vec := make([]int, len(F))
  i := 0

  for i < len(F) {
    // Check if numbers in `F` divides `n`
    if new(big.Int).Rem(nCopy, F[i]).Cmp(big.NewInt(0)) == 0 {
      // Append to `a` and increment the right index in `vec`
      a = append(a, F[i])
      vec[i] = ((vec[i] + 1) % 2)

      // Divide away the factor just added
      nCopy = nCopy.Div(nCopy, F[i])
    } else if F[i].Cmp(n) == 1 {
      break
    } else {
      i = i + 1
    }
  }

  if mult(a).Cmp(n) != 0 {
    return nil, nil, nil
  }

  return vec, a, nil
}

func GenerateNumber(N *big.Int, k *big.Int, j *big.Int) *big.Int {
  kN := new(big.Int).Mul(k, N)
  root := new(big.Int).Sqrt(kN)
  r := root.Add(root, j)

  return r
}

type group struct {
  point []int64
  r *big.Int
  r2 *big.Int
  factors []*big.Int
  vector []int
}

func solve(solution []string, G []group, N *big.Int) *big.Int {
  // Create left hand side and right hand side, non-multiplicated
  lhsSlice := make([]*big.Int, 0)
  rhsSlice := make([]*big.Int, 0)
  for i, v := range solution {
    if v == "1" {
      lhsSlice = append(lhsSlice, G[i].r)
      rhsSlice = append(rhsSlice, G[i].factors...)
    }
  }

  // Multiplicate the slice of lhs
  lhs := mult(lhsSlice)
  lhs = lhs.Mod(lhs, N)

  // Multiplicate the slice of rhs
  rhs := mult(rhsSlice)
  rhs = rhs.Sqrt(rhs)
  rhs = rhs.Mod(rhs, N)

  // Calculate gcd(|lhs - rhs|, N)
  gcd := new(big.Int).GCD(nil, nil, new(big.Int).Abs(new(big.Int).Sub(lhs, rhs)), N)
  return gcd
}

func main() {
  // Primes are 496469737391 and 511227320609
  N, _ := new(big.Int).SetString("253808893609854792191119", 10)

  F := Factorbase(2000)
  G := make([]group, 0)

  log.Println("Add random points to try...")

  // Add k, j-points to slice `P`
  for k := 1; k < 1500; k++ {
    for j := 1; j < 1500; j++ {
      p := []int64{int64(k), int64(j)}
      G = append(G, group{p, nil, nil, nil, nil})
    }
  }

  // Generate numbers based on the points from P
  for i := 0; i < len(G); i++ {
    r := GenerateNumber(
      N, // N
      big.NewInt(G[i].point[0]), // k
      big.NewInt(G[i].point[1]), // j
    )

    G[i].r = r
    G[i].r2 = new(big.Int).Exp(r, big.NewInt(2), N)
  }

  log.Println("Generating matrix...")
  // Add vector and factors to G iff it passes checks
  for i := 0; i < len(G); i++ {
    vector, factors, _ := FactorOverF(F, G[i].r2)

    dup := false
    for i := 0; i < len(G); i++ {
      if reflect.DeepEqual(G[i].vector, vector) == true {
        dup = true
        break
      }
    }

    if err != nil || dup == true {
      G[i].vector = nil
    }

    if dup == false {
      G[i].vector = vector
      G[i].factors = factors
    }
  }

  // Create a copy of G for reducing slice, and initalize a zero vector
  // for checking
  Gcopy := G[:0]
  zero := make([]int, len(F))
  for i := 0; i < len(zero); i++ {
    zero[i] = 0
  }

  // Reduce slice of degenerate cases, etc.
  for _, g := range G {
    if g.vector != nil && reflect.DeepEqual(g.vector, zero) != true {
      Gcopy = append(Gcopy, g)
    }
  }
  G = Gcopy

  log.Println("Matrix finished and printed!")

  // Write to file
  file, err := os.Create("input")
  if err != nil {
    log.Fatal("Cannot create file", err)
  }
  defer file.Close()

  fmt.Fprintf(file, strconv.Itoa(len(G)))
  fmt.Fprintf(file, " ")
  fmt.Fprintf(file, strconv.Itoa(len(F)))
  fmt.Fprintf(file, "\n")
  for i := 0; i < len(G); i++ {
    fmt.Fprintf(file, strings.Trim(strings.Join(strings.Fields(fmt.Sprint(G[i].vector)), " "), "[]"))
    fmt.Fprintf(file, "\n")
  }

  // Run GaussBin.exe
  cmd := exec.Command("GaussBin.exe", "input", "output")
  log.Printf("Running GaussBin.exe...")
  err = cmd.Run()
  log.Printf("Command finished with error: %v", err)

  // Read file line by line
  solutions := make([][]string, 0)
  outFile, err := os.Open("output")
  if err != nil {
    log.Fatal(err)
  }
  defer outFile.Close()

  scanner := bufio.NewScanner(outFile)
  for scanner.Scan() {
    solutions = append(solutions, strings.Fields(scanner.Text()))
  }

  if err := scanner.Err(); err != nil {
    log.Fatal(err)
  }

  solutions = solutions[1:]

  log.Println("Trying the solution vectors...")

  var factor *big.Int
  for i := 0; i < len(solutions); i++ {
    gcd := solve(solutions[i], G, N)
    if gcd.Cmp(big.NewInt(0)) != 0 && gcd.Cmp(big.NewInt(1)) != 0 {
      fmt.Println("============Found factor!============")
      factor = gcd
      break
    }
  }

  fmt.Println(factor)
}
\end{verbatim}

\begin{thebibliography}{9}
\bibitem{prime} 
Prime Counting function, \\
\texttt{http://mathworld.wolfram.com/PrimeCountingFunction.html}
\end{thebibliography}

\end{document}
